\documentclass[12pt, a4paper]{article}
\usepackage[utf8]{inputenc}
\usepackage[brazilian]{babel}
\usepackage[T1]{fontenc}
\usepackage{graphicx}
\usepackage{geometry}
\usepackage{color,soul}
\DeclareRobustCommand{\hlcyan}[1]{{\sethlcolor{cyan}\hl{#1}}}
\usepackage[document]{ragged2e}

\newcommand{\+}[1]{\ensuremath{\mathbf{#1}}} % comando para escrever matrizes em negrito \+A= A

\geometry{top=2.5cm, bottom=2.5cm, right=2.5cm, left=2.5cm}

\begin{document}
\noindent Institut national des postes et t�l�communications 
\noindent POO Avanc�e \hfill 2019\\
\noindent Prof. MAHMOUD EL HAMLAOUI \\
\bigbreak
\noindent Members: \newline
\noindent MARAOUI Souhail, RAIS Haythem, ZAGHLOUL Nabil, ELOMARI Mohamed \\ % Inserir o nome completo do aluno
\center{Mini-Project}
\center{\large{Hospital Management System}} % Inserir o t�tulo do artigo
\center{\textit{Report}} % Inserir o nome dos autores dos artigos
\justify

%%%%%%% Conte�do da resenha %%%%%%%%%%%


The project was divided into four applications that deal with different aspects of hospital management.
\bigbreak
 Queue System Management: \bigbreak
An application that manage hospital queues, where you enter your personnel information and your sickness symptoms, the application then would ask you to pay the cost of the visit (either with credit card or cash) and finally print you an invoice containing the doctor assigned to you(his name, his office number) and the (average) time you�d have to wait. \bigbreak
Doctor Software: \bigbreak
A software that display: \newline
-A list of patients the doctor treated, is treating, or will treat during his current shift. \newline
-For each patient : the patient information earlier and his whole sicknesses and treatments history. \newline
Once the doctor finish diagnosing the patient, he move to another panel where he can either: \newline
-Enter the patient sickness and the treatment he might need. \newline
-Decide if the patient need additional assistance or visits to recover. \newline
-Transfer to another doctor. \bigbreak
Emergency Calls Management: \bigbreak
A software where we can enter new calls for emergencies and manage already entered emergencies by assigning them an Ambulance and the staff to use, and once the emergency is flagged as resolved we assign the patient a new room with his doctor and nurses. \bigbreak
Hospital administration: \bigbreak
Where any admin can:  \newline
-Add, modify and delete staff members. \newline
-Add, modify and delete human and material resources.  \newline
-Add, modify and delete calls and view emergencies. \newline
Database design (conception): \newline
Patient ( IdPatient, FirstName, LastName, Gender, Age, Weight, Height, BloodType ) \newline
PatientHistory ( IdHistory, Sickness, Treatement, VisitDate, Amount, IdPatient, IdStaff ) \newline
Inpatient ( IdCare, IdStaff, IdRoom, IdPatient ) \newline
Room ( IdRoom, IsAvailable ) \newline
Queue ( IdQueue, Date, Status, IdPatient, IdStaff ) \newline
HumanResource ( IdStaff, FirstName, LastName, Username, Password, isAvailable, IdType, IdSpecialization ) \newline
Specialization ( IdSpecialization, Name, Fee ) \newline
HumanResourceType ( IdType, Name ) \newline
Log ( IdLog, IdStaff, History, ConcerningId) \newline
Call ( IdCall, FirstName, LastName, Date, IdEmergency ) \newline
Emergency ( IdEmergency, IdPatient, OccurenceDate, ResolutionDate, Location, Latitude, Longitude, Priority, Status ) \newline
 StaffUse ( IdEmergency, IdStaff ) \newline
Ambulance ( IdAmbulance, IsAvailable ) \newline
AmbulanceUse ( IdEmergency, IdAmbulance ) \newline



\end{document}